\documentclass[8pt,landscape]{article}

% PACKAGES
\usepackage[letterpaper, margin=0.25in]{geometry} % tighter margins
\usepackage{amsmath, amssymb}
\usepackage{multicol}
\usepackage{setspace}
\usepackage{sectsty}
\usepackage{verbatim}
% --- FORMATTING ---
\setstretch{0.9}
\setlength{\parindent}{0pt}
\setlength{\parskip}{0pt}
\allsectionsfont{\fontsize{8}{8}\selectfont}
% Reduce subsection spacing
\makeatletter
\renewcommand{\subsection}{\@startsection{subsection}{2}{0pt}%
    {0.2ex}% space before subsection
    {0.2ex}% space after subsection
    {\fontsize{8}{8}\bfseries}} % font size and bold
\makeatother
% DOCUMENT START
\begin{document}
\pagestyle{empty}
\begin{multicols}{3}
\subsection*{Discrete Distribution Families}
\subsection*{Bernoulli($p$)}
Binary trial (success/failure). \\
PMF: $P(X=x)=p^x(1-p)^{1-x}$, $x \in \{0,1\}$ \\
Mean: $p$, Var: $p(1-p)$ \\
R: \texttt{d/p/q/rbinom(n=1, p)}
\subsection*{Binomial($n,p$)}
successes in $n$ Bernoulli trials. \\
PMF: $P(X=x)=\binom{n}{x}p^x(1-p)^{n-x}$ \\
Mean: $np$, Var: $np(1-p)$ \\
R: \texttt{dbinom, pbinom, qbinom, rbinom}
\subsection*{Geometric($p$)}
Failures before 1st success. \\
PMF: $P(X=x)=(1-p)^x p$, $x \ge 0$ \\
Mean: $(1-p)/p$, Var: $(1-p)/p^2$ \\
R: \texttt{dgeom, pgeom, qgeom, rgeom}
\subsection*{NegBin($k,p$)}
Failures before $k$-th success. \\
PMF: $P(X=x)=\binom{x+k-1}{x}p^k(1-p)^x$ \\
Mean: $k(1-p)/p$, Var: $k(1-p)/p^2$ \\
R: \texttt{dnbinom, pnbinom, qnbinom, rnbinom}
\subsection*{Poisson($\lambda$)}
Counts in interval, rate $\lambda$. \\
PMF: $P(X=x)=\tfrac{\lambda^x e^{-\lambda}}{x!}$ \\
Mean: $\lambda$, Var: $\lambda$ \\
R: \texttt{dpois, ppois, qpois, rpois}
\subsection*{Continuous Distribution Families}
\subsection*{Uniform($a,b$)}
All values equally likely on $[a,b]$. \\
PDF: $f(x)=\tfrac{1}{b-a}, \; a \le x \le b$ \\
Mean: $(a+b)/2$, Var: $(b-a)^2/12$ \\
Skewness: 0 (symmetric) \\
R: \texttt{dunif, punif, qunif, runif} \\
\subsection*{Normal($\mu,\sigma^2$)}
Bell-shaped, $-\infty < \mu < \infty , \sigma^2 > 0 $\\
PDF: $f(x)=\tfrac{1}{\sqrt{2\pi\sigma^2}} e^{-(x-\mu)^2/(2\sigma^2)}$ \\
Mean: $\mu$, Var: $\sigma^2$ \\
Skewness: 0 (symmetric) \\
R: \texttt{dnorm, pnorm, qnorm, rnorm;} \\
\subsection*{Lognormal($\mu,\sigma^2$)}
If $\ln X \sim N(\mu,\sigma^2)$. \\
PDF: $f(x)=\tfrac{1}{x\sigma\sqrt{2\pi}} e^{-(\ln x-\mu)^2/(2\sigma^2)}$ \\
Mean: $e^{\mu+\sigma^2/2}$, Var: $(e^{\sigma^2}-1)e^{2\mu+\sigma^2}$ \\
Skewness: $(e^{\sigma^2}+2)\sqrt{e^{\sigma^2}-1}$ \\
R: \texttt{dlnorm, plnorm, qlnorm, rlnorm;} \\
\subsection*{Exponential($\lambda$)}
Time between Poisson events. \\
Positive RV, wait time, memoryless \\
($\lambda$) is average rate \\
($\beta$) is mean wait time \\
PDF: $f(x)=\lambda e^{-\lambda x}, x\ge0$ \\
Mean: $1/\lambda$, Var: $1/\lambda^2$ \\
Skewness: 2 \\
R: \texttt{dexp, pexp, qexp, rexp;} \\
\subsection*{Beta($\alpha,\beta$)}
On $[0,1]$, Bayesian priors. \\
Uniform distribution special case \\
The Gamma function ($\Gamma$)is a generalization \\
of the factorial function to non-integer numbers. \\
PDF: $f(x)=\tfrac{1}{B(\alpha,\beta)} x^{\alpha-1}(1-x)^{\beta-1}$ \\
Mean: $\alpha/(\alpha+\beta)$, Var: $\tfrac{\alpha\beta}{(\alpha+\beta)^2(\alpha+\beta+1)}$ \\
Skewness: $\frac{2(\beta-\alpha)\sqrt{\alpha+\beta+1}}{(\alpha+\beta+2)\sqrt{\alpha\beta}}$ \\
R: \texttt{dbeta, pbeta, qbeta, rbeta;} \\
\subsection*{Weibull($k,\lambda$)}
Lifetimes, survival, reliability. \\
Exponential family (when k = 1), longer you wait \\
Survival Analysis \\
PDF: $f(x)=\tfrac{k}{\lambda}(x/\lambda)^{k-1} e^{-(x/\lambda)^k}$ \\
Mean: $\lambda \Gamma(1+1/k)$, Var: $\lambda^2[\Gamma(1+2/k)-\Gamma(1+1/k)^2]$ \\
Skewness: $\frac{\Gamma(1+3/k)\lambda^3 - 3\mu\sigma^2 - \mu^3}{\sigma^3}$ \\
R: \texttt{dweibull, pweibull, qweibull, rweibull;} \\
\subsection*{Gamma($\alpha,\theta$)}
Waiting time for $\alpha$ events. \\
non-negative numbers \\
$\alpha$ is shape parameter \\
$\theta$ is scale parameter \\
PDF: $f(x)=\tfrac{1}{\Gamma(\alpha)\theta^\alpha} x^{\alpha-1} e^{-x/\theta}$ \\
Mean: $\alpha\theta$, Var: $\alpha\theta^2$ \\
Skewness: $2/\sqrt{\alpha}$ \\
R: \texttt{dgamma, pgamma, qgamma, rgamma;} \\
\subsection*{R: Computing Expected Value}
\subsection*{Discrete RV}
\texttt{X <- 0:2} \\
\texttt{p <- c(0.2,0.5,0.3)} \\
\texttt{EV <- sum(X*p)} \\
\subsection*{Continuous RV (numerical integration)}
\texttt{f <- function(x) 2*x} \\
\texttt{EV <- integrate(function(x) x*f(x), 0,1)\$value} \\
\subsection*{From sample (Monte Carlo)}
\texttt{samples <- rnorm(10000, mean=5, sd=2)} \\
\texttt{mean(samples)}  % approximate expected value
\subsection*{Define the PDF function}
\texttt{f <- function(x) { 2*x }} \\
\texttt{prob <- integrate(f, lower = 0.5, upper = 0.75)\$value \# Integrate over [0.5, 0.75]}
\subsection*{Conditional Distributions}
Let $X$ and $Y$ be two random variables.
\[
f_{X|Y}(x|y) =
\begin{cases}
\frac{P(X=x, Y=y)}{P(Y=y)}, & \text{discrete case} \\
\frac{f_{X,Y}(x,y)}{f_Y(y)}, & \text{continuous case, } f_Y(y) > 0
\end{cases}
\]
\texttt{Conditional dist is just a segment of marginal dist, then re-normalized to have an area under the curve equal to 1}
\texttt{If X and Y are independent, in continouse case,\[
f_{Y|X}(y) = f_Y(y)\] This means conditional PDF of Y and X is marginal PDF of Y. }
\subsection*{Random Sample}
\texttt{It is independent and identically distributed (iid). Each pair of observations are independent, and each observation comes from the same distribution.}
\subsection*{MLE}
\texttt{Great way to find estimators. Applied on multi and univariate. Relies on random sample of n observations. Mean, is a estimator for univariate, multivariate, linear regression}
\subsection*{Steps for MLE}
\texttt{1.Nature of variable (discrete or contin)}
\texttt{2.estimate the parameters of a theoretical distribution (eg $\lambda$ in a Poisson distribution)}
\texttt{3.Choose distribution: Normal, Exponential, Poisson, Binomial, etc.)}
\texttt{4.Play with the parameters for that family of distributions to find the one that would be most likely given our data and choose the corresponding parametric estimates)}
\texttt{5.To obtain these estimates, we use the likelihood function of our observed random sample.)}
\end{multicols}
\end{document}