\documentclass[8pt,landscape]{article}

% PACKAGES
\usepackage[letterpaper, margin=0.1in]{geometry}
\usepackage{amsmath, amssymb, geometry}
\usepackage{multicol}
\usepackage{setspace}
\usepackage{sectsty}
\usepackage{verbatim}
\usepackage{graphicx}
\usepackage{xcolor}
\usepackage{listings}
\usepackage{enumitem}
\usepackage{booktabs} % For better tables if needed

% Define a custom color for code highlighting and environment
\definecolor{myred}{RGB}{204, 0, 0} % A strong red
\definecolor{mygray}{RGB}{240, 240, 240} % Light gray for background
\setlist[itemize]{
    noitemsep,  % Removes vertical space between list items
    topsep=0pt,  % Removes space before the list
    partopsep=0pt, % Removes extra space when a list begins mid-paragraph
    parsep=0pt, % Removes space between paragraphs within an item
    leftmargin=*, % Sets the left margin to the minimum necessary
    labelsep=0.5em, % You can slightly adjust the space between the bullet and the text
}
% LISTINGS SETUP for R and Python
\lstset{
  basicstyle=\ttfamily\scriptsize\color{black},
  backgroundcolor=\color{mygray},
  frame=single,
  frameround=tttt,
  framesep=0pt,        % no padding between code and frame
  rulecolor=\color{black!20},
  rulesep=0pt,         % no space between rules
  breaklines=true,
  columns=fullflexible,
  showstringspaces=false,
  commentstyle=\color{gray},
  keywordstyle=\color{blue!80!black},
  stringstyle=\color{myred},
  breakatwhitespace=true,
  xleftmargin=0pt,     % no left margin
  xrightmargin=0pt,    % no right margin
  aboveskip=0pt,       % no space above listing
  belowskip=0pt,       % no space below listing
  abovecaptionskip=0pt,
  belowcaptionskip=0pt,
  framexleftmargin=0pt,
  framexrightmargin=0pt,
  framextopmargin=0pt,
  framexbottommargin=0pt
}

% FORMATTING
\setstretch{0.9} % Reduce line spacing
\setlength{\parindent}{0pt}
\setlength{\parskip}{0pt}

% Reduce section spacing and change color
\makeatletter
\renewcommand{\section}{\@startsection{section}{2}{0pt}%
    {0.1ex}% space before section
    {0.1ex}% space after section
    {\fontsize{8}{9}\bfseries\color{red}}} % section font: 8pt
\makeatother

% Reduce subsection spacing and make subsections blue
\makeatletter
\renewcommand{\subsection}{\@startsection{subsection}{2}{0pt}%
    {0.1ex}% space before subsection
    {0.1ex}% space after subsection
    {\fontsize{8}{9}\bfseries\color{blue}}} % Subsection font: 8pt
\makeatother

% Custom command for red-highlighted code/commands (for in-line use)
\newcommand{\code}[1]{\textcolor{myred}{\texttt{#1}}}

% Custom small text wrapper - ensuring 8pt for body text
\newcommand{\smalltext}[1]{
  {\fontsize{8}{9}\selectfont\sloppy #1\par}
}

% DOCUMENT START
\begin{document}
\fontsize{8}{9}\selectfont % Set the base font size to 8pt and line skip to 9pt for the entire document
\pagestyle{empty}
\begin{multicols}{3}

% \section{Ordinary Least-squares Regression}
% \subsection{Assumptions}
% \smalltext{
%     Response=Systematic Component+Random Component.
%     .$\epsilon$  is the Random Component under the following assumptions:
%     .each Yi is also assumed to be independent and normally distributed
%     .qq plot lying on the 45° degree dotted line. Standard Normal distribution
%     .Histogram of residuals bell-shaped form as in the Normal distribution.
%     .Homoscedasticity can be assessed via the diagnostic plot of residuals vs. fitted values. Funnel shapes indicates non-constant variance, i.e., heteroscedasticity.
%     .This assumption commonly gets violated in multiple linear regression and is called heteroscedasticity: the variance of the $\epsilon_i$s is not constant.
%     .OLS not suffice - Non-negative values.  and Binary outcomes (Success or Failure). and Count data. 
% }
\subsection{}
\smalltext{
    \textbf{Deterministic:}For each one of the values of the regressor X, there is a single value of Y.
    \textbf{Stochastic:} Each value of X has a probability distribution associated to Y.
    \textbf{Black-box Models:}is focused on optimizing predictions subject to a set of regressors with less attention on the internal model's process. 
    \textbf{Link function:}OLS regression models a continuous response $Y_i$  (a random variable) via its conditioned mean (or expected value) $\mu_i$ subject to $k$  regressors $X_{i,j}$. modelling the mean $\mu_i$ of a discrete-type response (such as binary or a count) is not straightforward. 
}


\begin{lstlisting}[language=R]
library(glmbb)
data(crabs)
crabs <- crabs |>  rename(n_males = satell) |>  dplyr::select(-y)
group_avg_width <- crabs |> mutate(intervals = cut(crabs$width, breaks = 10)) |> group_by(intervals) |> summarise(mean = mean(n_males), n = n()) 
poi_model <- glm(n_males ~ width, family = poisson, data = crabs)
\end{lstlisting}

\end{multicols}
\end{document}
