\documentclass[8pt,landscape]{article}

% PACKAGES
\usepackage[letterpaper, margin=0.1in]{geometry}
\usepackage{amsmath, amssymb, geometry}
\usepackage{multicol}
\usepackage{setspace}
\usepackage{sectsty}
\usepackage{verbatim}
\usepackage{graphicx}
\usepackage{xcolor}
\usepackage{listings}
\usepackage{enumitem}
\usepackage{booktabs} % For better tables if needed

% Define a custom color for code highlighting and environment
\definecolor{myred}{RGB}{204, 0, 0} % A strong red
\definecolor{mygray}{RGB}{240, 240, 240} % Light gray for background
\setlist[itemize]{
    noitemsep,  % Removes vertical space between list items
    topsep=0pt,  % Removes space before the list
    partopsep=0pt, % Removes extra space when a list begins mid-paragraph
    parsep=0pt, % Removes space between paragraphs within an item
    leftmargin=*, % Sets the left margin to the minimum necessary
    labelsep=0.5em, % You can slightly adjust the space between the bullet and the text
}
% LISTINGS SETUP for R and Python
\lstset{
  basicstyle=\ttfamily\scriptsize\color{black},
  backgroundcolor=\color{mygray},
  frame=single,
  frameround=tttt,
  framesep=0pt,        % no padding between code and frame
  rulecolor=\color{black!20},
  rulesep=0pt,         % no space between rules
  breaklines=true,
  columns=fullflexible,
  showstringspaces=false,
  commentstyle=\color{gray},
  keywordstyle=\color{blue!80!black},
  stringstyle=\color{myred},
  breakatwhitespace=true,
  xleftmargin=0pt,     % no left margin
  xrightmargin=0pt,    % no right margin
  aboveskip=0pt,       % no space above listing
  belowskip=0pt,       % no space below listing
  abovecaptionskip=0pt,
  belowcaptionskip=0pt,
  framexleftmargin=0pt,
  framexrightmargin=0pt,
  framextopmargin=0pt,
  framexbottommargin=0pt
}

% FORMATTING
\setstretch{0.9} % Reduce line spacing
\setlength{\parindent}{0pt}
\setlength{\parskip}{0pt}


% Reduce section spacing and change color
\makeatletter
\renewcommand{\section}{\@startsection{section}{2}{0pt}%
    {0.1ex}% space before section
    {0.1ex}% space after section
    {\fontsize{8}{9}\bfseries\color{red}}} % section font: 8pt
\makeatother

% Reduce subsection spacing and make subsections blue
\makeatletter
\renewcommand{\subsection}{\@startsection{subsection}{2}{0pt}%
    {0.1ex}% space before subsection
    {0.1ex}% space after subsection
    {\fontsize{8}{9}\bfseries\color{blue}}} % Subsection font: 8pt
\makeatother

% Custom command for red-highlighted code/commands (for in-line use)
\newcommand{\code}[1]{\textcolor{myred}{\texttt{#1}}}

% Custom small text wrapper - ensuring 8pt for body text
\newcommand{\smalltext}[1]{%
  {\fontsize{8}{9}\selectfont\sloppy #1\par}%
}

% DOCUMENT START
\begin{document}
\fontsize{8}{9}\selectfont % Set the base font size to 8pt and line skip to 9pt for the entire document
\pagestyle{empty}
\begin{multicols}{3}

\section{Linear Regresion - Lecture 1}
\subsection{EDA tells us}
\smalltext{
knowing the size of the data || examining distributions of all variables using graphical and numerical summaries || identifying missing values and potential outliers || beginning to discover relationships between variables
}
\subsection{Population vs. sample}

\begin{lstlisting}[language=R]
set.seed(561)
dat_s <- sample_n(dat, 1000, replace = FALSE) #Sample
#Get summary statistics
dat_tax_SLR = dat_s |> select(assess_val, BLDG_METRE) |> 
    gather() |>
    group_by(key) |>
    summarise(mean = mean(value, na.rm = TRUE), 
              max = max(value), 
              min = min(value), 
              median = median(value, na.rm = TRUE), 
              sd = sd(value, na.rm = TRUE))
\end{lstlisting}

\section{Simple Linear Regression (SLR)}
\subsection{The conditional expectation}
\smalltext{
  The conditional expected value is the best predictor given additional relevant information.
 SLR answers the question, given the size of property what can be the expected value?
}
\subsection{Assumptions}
\smalltext{
  Conditional expectation of the response is linearly related to the input variable and the line is the linear regression 
  || The random errors are independent and identically distributed: iid assumption
|| The random errors have all the same variance: $Var(\epsilon_i) = \sigma^2$
}

\smalltext{
$\text{Y}_i = \beta_0 + \beta_1\,\text{X}_i + \varepsilon_i$ \\
$\beta_0$ is the population intercept, and $\beta_1$ is the population slope parameter measuring how $\text{Y}$ changes with an associated change in $\text{X}$. \\
For each observation $i$, $\varepsilon_i$ represents the random error term. \\
\textbf{Intercept ($\beta_0$):} The average value of $Y$ when $X=0$ is $\beta_0$ \\
\textbf{Slope ($\beta_1$):} a one-unit increase in $X$ ($\text{X}$) is associated with an expected increase of $\beta_1$ units in $Y$ ($\text{Y}$). \\
\textbf{Error Term ($\epsilon_i$):} Any distributional assumption made about the error term also affect the random variable Y (if normal, Y also normal)
}

\section{Estimation of RL}
\subsection{Least squares estimation}
\smalltext{
Least Squares method minimizes the sum of the squares of the residuals. The residuals are the difference between the observed value of the response (${y}_i$) and the predicted value of the response  ($\hat{y}_i$)
as $r_i = y_i - \hat{y}_i$
}

\section{SLR with continuous variables}
\begin{lstlisting}[language=R]
SL_reg <- lm(col1 ~ col2, data = data_frame)
tidy_SL_reg <- tidy(SL_reg) #Get estimate, std.error, p value
lm(response ~ .,data= df) # uses all variables in the df, except the response, as predictors
lm(response ~ input - 1,data= df) #forces the estimated intercept to be 0.
beta_0_hat = SL_reg$coef[1] # Extract coefficients
beta_1_hat = SL_reg$coef[2]
\end{lstlisting}

\subsection{Estimated intercept $\beta_0$}
\smalltext{Not interested because, the intercept value does not make sense for the model. many statistical properties do not hold for models without an intercept}

\section{SLR with categorical predictors}
\smalltext{
  You have to use t-test, permutation test, or bootstrapping test to answer this kind of question. We use boxplots. X-axis is not numeric. 
  We can still use this variable in a Linear model. We can reproduce the results of a t-test using lm.
}
\begin{lstlisting}[language=R]
Condition_plot <- dat_s %>% ggplot(aes(Condition, assess_val)) + geom_boxplot()
t.test(assess_val~Condition,dat_s,var.equal=T)
lm_F <- lm(assess_val~Condition,dat_s)
tidy(lm_F)
\end{lstlisting}
\includegraphics[width=0.50\linewidth]{boxplot.png}

\subsection{The conditional expectation}
\begin{itemize}
    \item the best predictor of X with a Condition: 
    $E[Y \mid \text{Condition} = \text{Y}] = \mu_1$
    \item the best predictor of X without Condition: 
    $E[Y \mid \text{Condition} = \text{N}] = \mu_0$
    \item and a two-sample $t$-test tests the difference between group means!!
\end{itemize}
\smalltext{
\[
H_0 : \mu_1 = \mu_0, \quad \text{or equivalently} \quad H_0 : \mu_1 - \mu_0 = 0
\]
Since Condition is not numeric, we cant use in math formula, we instead use
The dummy variable:
\[
X_2 =
\begin{cases}
1 & \text{if Condition = Y},\\
0 & \text{if Condition = N}
\end{cases}
\]
\[
E[Y \mid X_2] = \beta_0 + \beta_2 X_2
\]
\begin{itemize}
    \item if Condition = N: \(\,E[Y \mid X_2 = 0] = \beta_0\)
    \item if Condition = Y: \(\,E[Y \mid X_2 = 1] = \beta_0 + \beta_2\)
\end{itemize}
Then
\[
\beta_2 = E[Y \mid X_2 = 1] - E[Y \mid X_2 = 0] = \mu_1 - \mu_0
\]
\[
H_0 : \beta_2 = 0 \text{ is the same as the null hypothesis from the two-sample } t\text{-test!}
\]
}
\subsection{The estimated intercept}
\smalltext{
  It is the sample version of the conditional expectation (mean of the reference group)
}

\subsection{The estimated slope}
\smalltext{
  It is the sample version of the difference of the conditional expectations (or group means)
}

\section{Uncertainty in the estimation - CI}
\smalltext{
  We can think that the estimates are a (good) guess about the population parameters based on our data. However, the values of the estimates depend on the random sample used to compute them:
}
\subsection{The standard errors}
\smalltext{
  The variation of these estimates from sample to sample is measured by their standard deviation, which has a special name: the standard error (SE).
  In practice, we have different ways to measure the sample-to-sample variation of the estimated coefficients: \\ 
(i). take multiple samples from the population and compute multiple estimates as we did above, then compute their SE (but this is not a realistic option). \\
(ii). use a theoretical result (this is what lm does) \\
(iii). use bootstrapping as you did in DSCI 552 for other quantities! \\
}

\section{Sampling Distribution}

\subsection{1-Bootstrapping}
\smalltext{
Bootstrapping refers to sampling from our original sample with replacement (also called resampling with replacement) to generate a many estimates and measure the sample-to-sample variation.
}
\begin{lstlisting}[language=R]
set.seed(0561)
lm_boot <- replicate(1000, {
  sample_n(dat_s, size = nrow(dat_s), replace = TRUE) %>% 
    lm(X~Y,data=.) %>% 
    .$coef 
})
lm_boot <- data.frame(boot_intercept = lm_boot[1,], boot_slope = lm_boot[2,]) 
boot_s <- dat_s |> 
    specify(assess_val~BLDG_METRE) |> 
    generate(reps = 1000, type = 'bootstrap') |> 
    calculate(stat = 'slope')
data.frame(coef_table %>% select(estimate),coef_table %>% select(std.error),
B_avg = lm_boot %>% summarize(intercept = mean(boot_intercept),
                               slope = mean(boot_slope)) %>% 
                     round(3)  %>% unlist(),
B_se = lm_boot %>% summarize(intercept = sd(boot_intercept),
                              slope= sd(boot_slope)) %>% 
                    round(3)  %>% unlist())
\end{lstlisting}

\subsection{Using theoretical results - CLT}
\smalltext{assumptions about the distribution of the error terms! || the conditional distribution of the error terms is Normal, and so is the conditional distribution of the response! || sample size is large || 
lm uses this theoretical result to compute p-values and confidence intervals!}

\subsection{Confidence Intervals}
\smalltext{}
\begin{lstlisting}[language=R]
tidy(lm_s, conf.int = TRUE)
quantile(slope_B,0.025)  %>% round(2)
quantile(slope_B,0.975) %>% round(2)
(slope_s + qnorm(0.025) * sd(slope_B)) %>% round(2)
(slope_s - qnorm(0.025) * sd(slope_B)) %>% round(2)
\end{lstlisting}

\subsection{Slope Coefficient Hypothesis}
\smalltext{
$H_0: \beta_1 = 0$ \\
$H_a: \beta_1 \neq 0$ \\
From the regression output we get the following: \\
Slope estimate: tidy\_SL\_reg -- estimate column  \\
p-value: tidy\_SL\_reg -- p-value column \\
If the p-value is far smaller than significance value $\alpha$ (0.05 usually): \\
$\text{p-value} < 0.05 \quad \Rightarrow \quad \text{Reject } H_0$ \\
Has is extremely strong statistical evidence that the slope coefficient differs from zero. We therefore conclude that col1 is a significant predictor of col2 in the population.
}
\subsection{The range problem}
\smalltext{
The linear model assumes that the relationship between $X$ and $E$[$Y$|$X$] is linear, which may or may not be true \\
Sometimes, there's a linear association only in part of the data range.
The linear model could still be useful when restricted to that specific range;
We need to exercise caution when using the model outside the range of the data, as the relationship between $X$  and $Y$ may differ significantly.
}
\section{Multiple Linear Regresion}
\smalltext{
$Y_i = \beta_0 + \beta_1 x_{i1} + \beta_2 x_{i2} + \cdots + \beta_p x_{ip} + \varepsilon_i$ \\
$\varepsilon_i \sim N(0, \sigma^2) \quad \text{for all } i$ \\
$\text{Cov}(\varepsilon_i, \varepsilon_j) = 0 \quad \text{for } i \neq j $ \\
The MLR model is no longer a line, but a hyperplane in a ($p$+1)-dimensional space. In three dimensions (two predictors), it is a plane. \\
}

\begin{lstlisting}[language=R]
fit <- lm(assess_val ~ BLDG_METRE*FIREPLACE, data= dat_s)
ML_reg <- lm(col1 ~ col2 + col3 + col4, data = data_frame)
library(GGally)
ggpairs(data = dataFrame[, c("col1","col2","col3","col4",)])
\end{lstlisting}

\subsection{"+" and "*" interaction in lm()}
\smalltext{
In an \texttt{lm()} formula, the \texttt{+} symbol adds a variable as a main effect only, meaning it includes the predictor in the model without forming any interaction terms.  
The \texttt{*} symbol expands to include both the main effects of the variables and their interaction term—for example, \texttt{x1 * x2} is equivalent to \texttt{x1 + x2 + x1:x2}.  
Thus, using \texttt{+} simply adds predictors, while using \texttt{*} fits a richer model that accounts for combined (interaction) effects between predictors.
}

\begin{lstlisting}[language=R]
# Examples in R:
lm(y ~ x1 + x2)      # main effects only
lm(y ~ x1 * x2)      # x1, x2, and interaction x1:x2
lm(y ~ x1:x2)        # interaction only
format(3.904359e-02, scientific = FALSE) #convert exponents to dec
\end{lstlisting}
\smalltext{
The bootstrap sampling distribution of the slope is approximately bell-shaped and symmetric, closely resembling a normal distribution. There are no major signs of skewness or heavy tails, indicating that the sampling distribution of the slope is well-approximated by the normal model commonly assumed in linear regression.
}
\end{multicols}
\end{document}