\documentclass[8pt,landscape]{article}

% PACKAGES
\usepackage[letterpaper, margin=0.1in]{geometry}
\usepackage{amsmath, amssymb}
\usepackage{multicol}
\usepackage{setspace}
\usepackage{sectsty}
\usepackage{verbatim}
\usepackage{graphicx}
\usepackage{xcolor}
\usepackage{listings}
\usepackage{xcolor}
\usepackage{enumitem}

\setlist{nosep, leftmargin=*}
\lstset{
  basicstyle=\ttfamily\footnotesize,
  backgroundcolor=\color{gray!10},
  frame=single,
  breaklines=true,
  columns=fullflexible,
  showstringspaces=false,
  keywordstyle=\color{blue},
  commentstyle=\color{gray},
  stringstyle=\color{orange}
}
% FORMATTING
\setstretch{0.9}
\setlength{\parindent}{0pt}
\setlength{\parskip}{0pt}

% Reduce subsection spacing and make subsections blue
\makeatletter
\renewcommand{\subsection}{\@startsection{subsection}{2}{0pt}%
    {0.2ex}% space before subsection
    {0.2ex}% space after subsection
    {\fontsize{8}{8}\bfseries\color{blue}}} % font size, bold, blue color
\makeatother
% Custom small text wrapper
\newcommand{\smalltext}[1]{%
  {\fontsize{8}{7}\selectfont\sloppy #1\par}%
}

% DOCUMENT START
\begin{document}
\pagestyle{empty}
\begin{multicols}{3}
\subsection*{Code}
\smalltext{critical value $Z$ for a 99\% confidence interval using the qnorm() function. For a two-sided confidence interval, a 99\% confidence level means that $\alpha = 1 - 0.99 = 0.01$, and we are interested in the value $Z$ such that the area to the left is $1 - \alpha/2 = 1 - 0.01/2 = 0.995$.}
\begin{lstlisting}[language=R]
#critical value Z
qnorm(0.995) #give 2.58
\end{lstlisting}

\smalltext{
    Based on the critical value calculated in the previous step, $\text{z\_99} = 2.58$, the formula for the 99\% confidence interval is:$$\text{Confidence Interval} = \text{Point Estimate} \pm \underbrace{\, 2.58 \, \times \, \text{SE}}_{\text{MoE}}$$The critical $Z$-value, $\mathbf{Z_{1 - \alpha/2}}$, is replaced by $\mathbf{2.58}$, which corresponds to the $\mathbf{99\%}$ confidence level ($\alpha = 0.01$). This value is the multiplier for the Standard Error ($\text{SE}$) to calculate the Margin of Error ($\text{MoE}$).
}

\end{multicols}
\end{document}